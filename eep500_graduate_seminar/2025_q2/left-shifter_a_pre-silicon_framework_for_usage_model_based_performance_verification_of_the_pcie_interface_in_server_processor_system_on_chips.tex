\documentclass{article}
% Template from https://www.overleaf.com/project/67ce344378c4ecdee5f2fef9

% Language setting
% Replace `english' with e.g. `spanish' to change the document language
\usepackage[english]{babel}

% Set page size and margins
% Replace `letterpaper' with`a4paper' for UK/EU standard size
\usepackage[letterpaper,top=2cm,bottom=2cm,left=3cm,right=3cm,marginparwidth=1.75cm]{geometry}

% Useful packages
\usepackage{amsmath}
\usepackage{graphicx}
\usepackage[colorlinks=true, allcolors=blue]{hyperref}

\title{Left-Shifter: A Pre-silicon Framework for Usage Model
Based Performance Verification of the PCIe Interface
in Server Processor System on Chips}
\author{Enrique Antunano}

\begin{document}
\maketitle

\section{Response}
The paper focuses on the development of a PCIe pre-silicon development framework that should be used during Compute System on Chip (SoC) silicon development to characterize PCIe performance, identify design pitfalls, and provide design rectification solutions prior to spinning a novel Compute SoC. The authors identify that currently, there are no pre-silicon PCIe development solutions on the market that can verify whether a Compute SoC meets the PCIe bandwidth performance requirements. Designers can only characterize performance and identify silicon bugs through post-silicon testing, which is expensive and time-consuming. Design of the PCIe framework involves four key components that together emulate the PCIe interface and Compute SoC. First, the framework models the different endpoints for which PCIe interfaces are typically used. Then, the framework provides a software stack model which manages system initialization, I/O interactions of the modeled endpoints, and resource need management of the I/O endpoints. Third, a test case for each endpoint usage model, and fourth scripts for calculation PCIe bandwidth limits and PCIe page table generation. 

Metrics provided by the framework include bandwidth utilization from the endpoint's perspective. Read request latency (ns) from the endpoint device and also from the root port to the SoC interface edge. Finally, the framework measures the write request latency (ns) from the root port to the SoC interface edge. Using the framework, the research team was able to identify bandwidth utilization below expectations for three different test cases, and fix the under bandwidth utilization prior to silicon fabrication. Using the framework, the total development effort involved with developing the use-case specific software stack and silicon bare-metal design was two person years. From there, the emulation tests for the different endpoint scenarios were 1-2 hours, which is negligible with respect to the entire design cycle. 

Currently at my work, there are no use case scenarios. My team has implemented PCIe and DMA interfaces in FPGAs to interface with the board processor, but the silicon itself was designed by the FPGA and processor manufacturers respectively. Although not directly related to my work, this paper provided a thorough breakdown of how to test, verify, and characterize the performance of a PCIe interface across different endpoint test cases. Previously, I had no understanding on how to characterize the performance of a PCIe interface and if asked to do so at work, I would reference this paper again for either pre-silicion or post-silicon characterization. At a quick glance, I would look at whether there are ways to reduce the person-year development time from two years, as this would further reduce the cost of Compute SoC silicon development.

\nocite{*}

\bibliographystyle{ieeetr}
\bibliography{left-shifter_a_pre-silicon_framework_for_usage_model_based_performance_verification_of_the_pcie_interface_in_server_processor_system_on_chips.bib}

\end{document}