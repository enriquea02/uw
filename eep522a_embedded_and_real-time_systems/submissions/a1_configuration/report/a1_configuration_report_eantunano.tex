\documentclass[journal]{IEEEtran}
\usepackage{graphicx}
\usepackage{hyperref}
\usepackage{listings}
% Path is relative to a1_configuration_report_eantunano.tex
\graphicspath{ {./images/}}

\begin{document}

    \title{Raspberry Pi 4 Embedded Development Environment}

    \author{Enrique~Antunano\\UW\\enriantu@uw.edu}

    \markboth{EEP522A~Embedded and~Real~Time~Systems, A1~Configuration, January~2025}{}

    \maketitle

    \begin{abstract}

    \end{abstract}
    The purpose of the A1 Configuration assignment is to set up the Raspberry Pi 4 and the embedded development environment that will be used to interact with the machine. 
    \section{Introduction}

    \section{Methods}

    \section{Results}

    \section{Discussion}

    \section{Conclusion}

    \nocite{*}
    \newpage

    \bibliographystyle{ieeetr}
    \bibliography{a1_configuration_citations.bib}

    \newpage
    \section{Acknowledgements}
    The author would like to thank IEEE for providing the LaTeX template used in this document and Michael Shell of the Georgia Institute of Technology for putting together the template.

    \appendices
    \section{Install OS onto Raspberry Pi 4}
    \subsection{Raspberry Pi Imager}

    The Raspberry Pi Imager assists with installing the OS onto a storage device, which can then be installed onto the Raspberry Pi 4.

    \begin{enumerate}
        \item Hardware checklist
        
        \begin{itemize}
            \item Linux machine that can store the OS image onto microSD card \newline
                \emph{Distribution: POP!\_OS 22.04 LTS (Debian)}
            \item SD card reader
            \item minimum 16GB microSD card
        \end{itemize}
    
        \item Install the Raspberry Pi Imager from the Linux terminal \newline
        
        Type out the command below and hit \emph{Enter}. \newline

        \begin{lstlisting}[frame=single]
            sudo apt install rpi-imager
        \end{lstlisting}

        \emph{NOTE: POP!\_OS 22.04 LTS includes the Raspberry Pi Imager in its package manager. The package manager has Raspberry Pi Imager v1.7.2, which these instructions will use. If your machine's OS does not include the Raspberry Pi Imager in its package manager, then install the {\href{https://www.raspberrypi.com/software/}{Raspberry Pi Imager}}.} \newline

        Following the figure below, select \emph{y} when prompted. \newline

        \includegraphics[width=2.5in]{sudo_apt_install_rpi_imager.png}

        After selecting \emph{y}, the terminal window should match the figure below after installation has completed. \newline

        \includegraphics[width=2.5in]{sudo_apt_install_rpi_imager_download_progress.png}

        \item Run Raspberry Pi Imager by opening a Linux terminal, typing the following, and selecting \emph{Enter}
        
        \begin{lstlisting}[frame=single]
            rpi-imager
        \end{lstlisting}

        \item Select \emph{CHOOSE OS}

        \includegraphics[width=2.5in]{choose_os.png}
        
        \item Select \emph{Raspberry Pi OS (Legacy,32-bit)}
        
        \includegraphics[width=2.5in]{os_available.png} \newline

        A close-up figure  of the debian that will be installed onto the microSD card. \newline

        \includegraphics[width=2.5in]{rp_os_debian_bullseye.png}
        
        \item Plug-in microSD card into the SD card reader
        \item Plug-in SD card reader into Linux machine

        \item Select \emph{CHOOSE STORAGE}
        
        \includegraphics[width=2.5in]{choose_storage.png}
        
        \item Select storage media. \newline
        HeydayCardReader was my SD card reader. \newline
        31.9 GB represents the size of the microSD card

        \includegraphics[width=2.5in]{sd_card_reader_name.png}

        \item Select the storage icon at the bottom-right of the screen \newline
        
        The Raspberry Pi OS image will be configured here to ease setup on the Raspberry Pi 4

        \includegraphics[width=2.5in]{choose_settings_icon.png}

        The following advanced options screen will appear. \newline

        \includegraphics[width=2.5in]{advanced_options_initial_screen.png}

        \item Set the hostname to be \emph{raspberrypi4}
        \item Select the \emph{Enable SSH} option and select \emph{Use password authentication}
        \item Select the \emph{Set username and password} option and fill in a \emph{username} and \emph{password} of your choosing.
        
        \includegraphics[width=2.5in]{advanced_options_hostname_enable-ssh_user_pass.png}

        \item Select the \emph{Configure wireless LAN} option and fill out the \emph{SSID} and \emph{Password}
        
        \item Select \emph{Set locale settings} and set \emph{Time Zone} to your local time and the \emph{keyboard layout} to your preferred layout
        
        \includegraphics[width=2.5in]{advanced_options_ssid_locale.png}

        \item Select \emph{Play sound when finished} and leave the other two options as enabled. Finally, select \emph{Save}.
        
        \includegraphics[width=2.5in]{advanced_options_misc_settings.png}

        \item Select \emph{WRITE} 
        
        \includegraphics[width=2.5in]{choose_settings_icon.png}

        \item Select \emph{YES} when prompted to continue despite all the data on the microSD card being erased.
        
        \includegraphics[width=2.5in]{clear_sd_card_warning.png}

        \item Type in password when asked to authenticate user
        
        \includegraphics[width=2.5in]{authenticate_password.png}

        \item Wait for SD write to completed
        
        \includegraphics[width=2.5in]{sd_card_write.png}

        \item Wait for SD write verification to completed
        
        \includegraphics[width=2.5in]{sd_card_verify.png}

        \item Click \emph{continue}. The following message will appear after the SD card has successfully been programmed with the Raspberry OS Bullseye debian.
        
        \includegraphics[width=2.5in]{end_write_message.png}

        \item Safely eject your SD card reader from the Linux machine 

    \end{enumerate}

    \subsection{Initial Raspberry Pi 4 Setup}

    \section{Figures and Tables}

\end{document}