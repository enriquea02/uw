\documentclass[journal]{IEEEtran}
\usepackage{graphicx}
\usepackage{hyperref}
\usepackage{listings}
% Path is relative to a2_configuration_report_eantunano.tex
\graphicspath{ {./images/}}

\begin{document}

    \title{Raspberry Pi 4 Exploration}

    \author{Enrique~Antunano\\University~of~Washington\\enriantu@uw.edu\\https://github.com/enriquea02/uw}

    \markboth{EEP522A~Embedded and~Real~Time~Systems, A2~Configuration, February~2025}{}

    \maketitle

    \begin{abstract}
    
    \end{abstract}
    \section{Introduction}

    For this assignment I am trying to characterize the Raspberry Pi 4 specifically for the purposes of a laser projector box. The laser projector box, I would like to would accept a wireless transmission
    \section{Methods}

    Laser projector box need
    \section{Results}
    \section{Discussion}
    CHALLENGES I RAN INTO: There's way too much that I need to find out!!!! I don't work with C and I don't like it, so writing code for this assignment has been difficult.

    INSIGHT FOR LIMITED OR MASS PRODUCTION: \newline
    I would not use the Raspberry Pi 4 for pass production since it's likely too powerful and expensive for my purposes. 
    Going on YouTube, most hobbyists use an Arduino to run their laser projector, which is 48\% more expensive. 
    This is based on a quick Google search of an Arduino Uno $23.50, while the Raspberry Pi 4 is $35.00. 
    Now, this suggestion is for small rates of production. For mass production, I would try to push for a smaller, cheaper, less power intensive device than even the Arduino Uno.

    POINTS FOR INSIGHTS AND THOUGHTS ON HOW NEW TOOLS WILL EFFECT EMBEDDED SYSTEMS


    \section{Conclusion}


    \nocite{*}
    \newpage

    \bibliographystyle{ieeetr}
    \bibliography{a2_configuration.bib}

    \newpage
    \section{Acknowledgements}

    \appendices
    \section{Characterization Documentation}
    This appendix section covers all the characterization documentation that was explored as part of my determination of whether the Raspberry Pi 4 would have a sufficient amount of material for quickly developing a prototype of a
    Laser Projector box, as part of my EPP422 final project.
    
    The determination comes down to the Raspberry Pi 4's hardware limitations, OS, associated software and pin libraries, documentation of the Raspberry Pi 4 and potential peripherals, and software tools that 
    will help with my development process.

    \subsection{Characterization Questions}
    This appendix section defines the characterization scope during the exploration phase. Not all questions will be answered, but answering a majority of the questions will help with the determination whether
    the Raspberry Pi 4 will be sufficient for the final project.
    \begin{enumerate}
        \item What SoC is on the target board? What version and manufacturer of the SoC?
        \item What features are present on the board/SoC?
        \item What size caches are present in the Raspberry Pi?
        \item What is the pin layout on this specific version of the target board?
        \item How does the board boot? What is the process?
        \item What software libraries are available for interfacing with the pins?
        \item What is the performance of the GPIO pins for PWM control?
        \item What is the latency for setting a GPIO pin from command to physical output change. i.e., 90\% of pin voltage rating)
        \item How well documented is each pin library and how easy is it to use?
        \item How much memory is available on the board?
        \item What is the maximum memory configuration?
        \item How many types of memory are on the board?
        \item What is the performance of the memory?
        \item What is the compiler name and version?
        \item What is required to add an on-switch to the Raspberry Pi?
        \item What is the interrupt latency for the board?
        \item How much time does it take to copy 1 KB, 1 MB, and 1GB in bytes, half words, and words in RAM?
        \item How much time does it take to copy 1 KB, 1 MB, and 1GB in bytes, half words, and words on the
        filing system?
        \item Determine the speed of integer arithmetic with a benchmark.
        \item Are there image processing libraries?
        \item How much memory might the image processing library require?
        \item What is the CPU time for an image processing task? Are there benchmarks available?
        
    \subsection{Characterization Commands}

    This command generates a human-readable listing of the total RAM on the Raspberry Pi 4.
    \begin{lstlisting}[frame=single]
        free -h
    \end{lstlisting}


    The Linux OS will output the following memory information. \newline
    
    \includegraphics[width=2.5in]{rpi4_ram.png}

    \end{enumerate}
    \section{Figures and Tables}
    \begin{table}[h!] 
        \begin{center}
          \caption{Raspberry Pi 4 Official Documentation Feature-set.}
          \label{tab:table1}
          \begin{tabular}{l|l} % <-- Alignments: 1st column left, 2nd middle and 3rd right, with vertical lines in between
            \textbf{Feature} & \textbf{Details}\\
            \hline
            SoC & BCM2711 Broadcom Quad-core ARM Cortex-A72 64-bit SoC @ 1.5GHz\\
            Memory & 1GB LPDDR4-2400 SDRAM \\
            Caches & 32kB data + 48kB instruction L1 cache per core. 1MB L2 cache.\\
            Multimedia & H.265 (4Kp60 decode)\\
            Multimedia & H.264 (1080p60 decode,1080p30 encode)\\
            Multimedia & OpenGL ES, 3.0 graphics\\
            I/O & PCIe bus\\
            I/O & onboard Ethernet port\\
            I/O & 1 x Display Serial Interface (DSI) port\\
            I/O & 1 x Camera Serial Interface (CSI) port\\
            I/O & 6 x I2C ports\\
            I/O & 6 x UART (muxed with I2C)\\
            I/O & 5 x SPI\\
            I/O & 2-lane MIPI Dual HDMI video output\\
            I/O & 2-lane MIPI Composite video output\\
            I/O & 40-pin GPIO Header\\
            I/O & 2 x USB 3.0 ports\\
            I/O & 2 x USB 2.0 ports\\
            Audio/Video & 4-pole stereo audio and composite video port\\
            Wireless Module & Dual channel 2.4/5.0 GHz IEEE 802.11ac wireless\\
            Wireless Module & Bluetooth 5.0\\
            Wireless Module & Bluetooth Low Energy (BLE)\\
            Input power & 5.1V/3A DC via USB-C connector\\
            Operating Temp. & 0 - 500C
          \end{tabular}
        \end{center}
      \end{table}

      \subsection{Figure 1 Raspberry Pi 4 40-pin Header}

      \includegraphics[width=2.5in]{rpi4b_40pin_header.png}

      \subsection{Figure 2 Raspberry Pi 4 GPIO Pins}

      \includegraphics[width=2.5in]{rpi4_40pin_header_gpio_numbering.png}

      \subsection{Figure 3 Alternative Pin Functions}

      \includegraphics[width=2.5in]{alternative_gpio_pin_funcs.png}
    

\end{document}