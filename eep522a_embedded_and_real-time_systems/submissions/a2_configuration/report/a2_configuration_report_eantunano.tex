\documentclass[journal]{IEEEtran}
\usepackage{graphicx}
\usepackage{hyperref}
\usepackage{listings}
% Path is relative to a2_configuration_report_eantunano.tex
\graphicspath{ {./images/}}

\begin{document}

    \title{Raspberry Pi 4 Exploration}

    \author{Enrique~Antunano\\University~of~Washington\\enriantu@uw.edu\\https://github.com/enriquea02/uw}

    \markboth{EEP522A~Embedded and~Real~Time~Systems, A2~Configuration, February~2025}{}

    \maketitle

    \begin{abstract}
    
    \end{abstract}
    \section{Introduction}

    For this assignment I am trying to characterize the Raspberry Pi 4 specifically for the purposes of a laser projector box. 
    The purpose of this paper is to develop a baseline for the Raspberry Pi 4 and determine whether it is sufficient for my project needs.
    For my project, I envision a laser projector box that can accept images and gifs through a wireless connection to a phone or other device. 
    Then the device will auto-load the image set and project them onto a wall. 
    Additionally, the device will be constantly connected to a power source and will not have a strict size and weight limit.

    \section{Methods}
    To meet the goal of developing a \emph{theoretical} baseline of capabilities, so that I can determine whether what I have available will be sufficient for my project goals, 
    I have created a list of characterization questions I intend to explore for my project.
    Not all questions will be answered, but the intent is that a majority of them would be answered and contain supporting material.
    Refer to Appendix A, sub-appendix A \emph{Characterization Questions} for the list of characterization questions that I attempted to answer as part of my project base lining.
    Device characterization was conducted on a Raspberry Pi 4 running the Raspbian OS with the Bullseye Debian.

    \subsection{Board/Layout Information}
    Questions \emph{1} and \emph{2} were pulled from the official Raspberry Pi documentation on their webpage and from Negi's \emph{Raspberry Pi 4 GPIO Pinout and Specifications (Updated)}.
    To help answer question \emph{3}, Appendix A, sub-appendix B \emph{Characterization Commands} covers the command necessary to poll the Raspberry Pi 4 for its available memory.
    Moving to question 4, the pin layout is covered in Appendix B, sub-appendix B \emph{Figures and Tables} Figures 1-3. 
    The information was gathered from Raspberry Pi 4's official documentation and other unaffiliated websites.
    Regarding question 5, refer to Figure 4 for the board boot process of the BCM2711 SoC.

    \subsection{GPIO Information}
    \section{Results}
    Answering questions \emph{1} to \emph{3}, refer to the table I compiled in Appendix B \emph{Figures and Tables}, Table 1. 
    The table summarizes the SoC, version, and manufacturer for the Raspberry Pi 4.
    Additionally, it covers the features present on the board, such a PWM hardware pins, SPI pins, and the encoding/decoding schemes supported by the Raspberry Pi 4 board.
    Finally, for question \emph{3}, the Raspberry Pi 4 official documentation listed out the cache and memory sizes this board.
    After SoC power up, the boot flow for the ROM is to read the EEPROM one-time programmable memory to determine which source the Raspberry Pi 4 will boot from.
    Assuming boot from a GPIO pin or second stage SPI boot loader is set, then the device will load from the SD memory slot and continue as normal with its boot. 
    \section{Discussion}
    I think the Raspberry Pi 4 L1 and L2 cache sizes will be sufficient for my purposes of on-board image processing. 
    The Raspberry Pi 4 will need to store intermediate values in the cache and memory as it conducts its matrix operations to covert a set of images and/or gifs into a matrix of floating points numbers.
    After the image processing, the Raspberry Pi 4 can store the data into its non-volatile memory.
    From non-volatile memory, the Raspberry Pi 4 can then switch to transferring the matrix into commands that will either toggle power to the laser diode and drive motor controls to the mirror(s) in the laser box. 
    
    As an interesting aside, one source listed the Raspberry Pi 4 operating temperature as 0\-500C.
    The upper range value appears unrealistic, and I doubt value. 
    For my purposes, the laser projector box will run indoors at ambient temperature, thus I am not worried about the operating temperature range.

    Regarding boot flow, I intend to boot the Raspberry Pi 4 from the SD card each time. 
    So this information is not actually useful for my final project, but it is interesting.
    If I were to switch to a mass production device (1000+ devices), then I would look into changing where the target device boots its image from.
    For mass production, it is expensive to boot from an SD card and inconvenient for the end-user to also purchase an SD-card to go along with their laser projector.
    
    As for the pin layout and available GPIO pins, it was interesting to find out that there were several pints available for PWM outputs.Futhermore, the PWM pins were further broken down to software and hardware set pins, which I think could be very useful.
    For the purposes of my laser controller box, it is likely that a hardware set PWM pin may be sufficient for the laser.
    Meanwhile, the complex software controlled PWMs may be required for driving the motors that rotate the mirrors.
    My statement may represent an abundant amount of naivety, but from reading of Pi forums, specifically the \emph{Hardware or Software?} forum, my decision to use the hardware PWM for light diodes and software PWM for motor controllers makes sense.
    With a software PWM, the laser diode may get stuck for a perceptible amount of time.
    The hardware PWMs have a limited frequency and duty cycle set, but is independent of the Raspberry Pi processor.
    Limited PWM flexibility for a laser is not critical because realisitically the laser is likely to be set to just \emph{ON} or \emph{OFF}.
    On the other hand, the motors will require precise control which can be achieved easily by a software PWM.
    Finally, physically moving the mirrors and the inertia from the mirrors provides a margin for error regarding glitch-less control.
    This comes from my naive understanding t 

    CHALLENGES I RAN INTO: There's way too much that I need to find out!!!! I don't work with C and I don't like it, so writing code for this assignment has been difficult.

    INSIGHT FOR LIMITED OR MASS PRODUCTION: \newline
    I would not use the Raspberry Pi 4 for pass production since it's likely too powerful and expensive for my purposes. 
    Going on YouTube, most hobbyists use an Arduino to run their laser projector, which is 48\% more expensive. 
    This is based on a quick Google search of an Arduino Uno $23.50, while the Raspberry Pi 4 is $35.00. 
    Now, this suggestion is for small rates of production. For mass production, I would try to push for a smaller, cheaper, less power intensive device than even the Arduino Uno.

    POINTS FOR INSIGHTS AND THOUGHTS ON HOW NEW TOOLS WILL EFFECT EMBEDDED SYSTEMS

    On the a2\_configuration.pdf, it lists the expected time to complete this project as 5 to 6 hours.
    Time spent for me was closer to 12-15 hours and I only did 30-40\% of my goal.
    My last C coding experience was during my undergraduate education several years ago, and I did not enjoy C as a coding language.
    For this assignment, I struggled to write my characterization functions since much of my time was spent understanding and writing C.
    Initially I planned a whole characterization test suite/library, but was only able to complete one function reliably, my RAM write speed test function.
    The subsequent function I wrote, a file system access test, caused my Raspberry Pi 4 to crash.
    There is a strong likelihood that I accessed the file system incorrectly or stressed the device, but I currently do not have the time available to debug the root cause using my current skill-set.  


    \section{Conclusion}
    Completing the initial exploration phase of this project, I am actually left with more questions than answers.
    This assignment has illuminated how little I understood about my project.
    Conversely, it has provided an initial guide on what challenges are likely to be the hardest to resolve before the project presentation deadline.

    \nocite{*}
    \newpage

    \bibliographystyle{ieeetr}
    \bibliography{a2_configuration.bib}

    \newpage
    \section{Acknowledgements}
    The author would like to thank the University of Washington Electrical and Computer Engineering Department for providing Raspberry Pi kits.

    The author would like to thank IEEE for providing the LaTeX template used in this document and Michael Shell of the Georgia Institute of Technology for putting together the template.

    The author would like to acknowledge the open-source community for their work on the documentation and source code associated with the WiringPi library. 

    \appendices
    \section{Characterization Documentation}
    This appendix section covers all the characterization documentation that was explored as part of my determination of whether the Raspberry Pi 4 would have a sufficient amount of material for quickly developing a prototype of a
    Laser Projector box, as part of my EPP422 final project.
    
    The determination comes down to the Raspberry Pi 4's hardware limitations, OS, associated software and pin libraries, documentation of the Raspberry Pi 4 and potential peripherals, and software tools that 
    will help with my development process.

    \subsection{Characterization Questions}
    This appendix section defines the characterization scope during the exploration phase. Not all questions will be answered, but answering a majority of the questions will help with the determination whether
    the Raspberry Pi 4 will be sufficient for the final project.
    \begin{enumerate}
        \item What SoC is on the target board? What version and manufacturer of the SoC?
        \item What features are present on the board/SoC?
        \item What size caches are present in the Raspberry Pi?
        \item What is the pin layout on this specific version of the target board?
        \item How does the board boot? What is the process?
        \item What software libraries are available for interfacing with the pins?
        \item What is the performance of the GPIO pins for PWM control?
        \item What is the latency for setting a GPIO pin from command to physical output change. i.e., 90\% of pin voltage rating)
        \item How well documented is each pin library and how easy is it to use?
        \item How much memory is available on the board?
        \item What is the maximum memory configuration?
        \item How many types of memory are on the board?
        \item What is the performance of the memory?
        \item What is the compiler name and version?
        \item What is required to add an on-switch to the Raspberry Pi?
        \item What is the interrupt latency for the board?
        \item How much time does it take to copy 1 KB, 1 MB, and 1GB in bytes, half words, and words in RAM?
        \item How much time does it take to copy 1 KB, 1 MB, and 1GB in bytes, half words, and words on the
        filing system?
        \item Determine the speed of integer arithmetic with a benchmark.
        \item Are there image processing libraries?
        \item How much memory might the image processing library require?
        \item What is the CPU time for an image processing task? Are there benchmarks available?
        
    \subsection{Characterization Commands}

    This command generates a human-readable listing of the total RAM on the Raspberry Pi 4.
    \begin{lstlisting}[frame=single]
        free -h
    \end{lstlisting}


    The Linux OS will output the following memory information. \newline
    
    \includegraphics[width=2.5in]{rpi4_ram.png}

    \end{enumerate}
    \section{Figures and Tables}
    \begin{table}[h!] 
        \begin{center}
          \caption{Raspberry Pi 4 Official Documentation Feature-set.}
          \label{tab:table1}
          \begin{tabular}{l|l} % <-- Alignments: 1st column left, 2nd middle and 3rd right, with vertical lines in between
            \textbf{Feature} & \textbf{Details}\\
            \hline
            SoC & BCM2711 Broadcom Quad-core ARM Cortex-A72 64-bit SoC @ 1.5GHz\\
            Memory & 1GB LPDDR4-2400 SDRAM \\
            Caches & 32kB data + 48kB instruction L1 cache per core. 1MB L2 cache.\\
            Multimedia & H.265 (4Kp60 decode)\\
            Multimedia & H.264 (1080p60 decode,1080p30 encode)\\
            Multimedia & OpenGL ES, 3.0 graphics\\
            I/O & PCIe bus\\
            I/O & onboard Ethernet port\\
            I/O & 1 x Display Serial Interface (DSI) port\\
            I/O & 1 x Camera Serial Interface (CSI) port\\
            I/O & 6 x I2C ports\\
            I/O & 6 x UART (muxed with I2C)\\
            I/O & 5 x SPI\\
            I/O & 2-lane MIPI Dual HDMI video output\\
            I/O & 2-lane MIPI Composite video output\\
            I/O & 40-pin GPIO Header\\
            I/O & 2 x USB 3.0 ports\\
            I/O & 2 x USB 2.0 ports\\
            Audio/Video & 4-pole stereo audio and composite video port\\
            Wireless Module & Dual channel 2.4/5.0 GHz IEEE 802.11ac wireless\\
            Wireless Module & Bluetooth 5.0\\
            Wireless Module & Bluetooth Low Energy (BLE)\\
            Input power & 5.1V/3A DC via USB-C connector\\
            Operating Temp. & 0 - 500C
          \end{tabular}
        \end{center}
      \end{table}

      \subsection{Figure 1 Raspberry Pi 4 40-pin Header}

      \includegraphics[width=2.5in]{rpi4b_40pin_header.png}

      \subsection{Figure 2 Raspberry Pi 4 GPIO Pins}

      \includegraphics[width=2.5in]{rpi4_40pin_header_gpio_numbering.png}

      \subsection{Figure 3 Alternative Pin Functions}

      \includegraphics[width=2.5in]{alternative_gpio_pin_funcs.png}

      \subsection{Figure 4 Board Boot Process}

      \includegraphics[width=2.5in]{board_boot.png}
    

\end{document}