\documentclass[journal]{IEEEtran}
\usepackage{graphicx}
\usepackage{hyperref}
\usepackage{listings}
% Path is relative to final_project_report_eantunano.tex
\graphicspath{ {./images/}}

\begin{document}

    \title{Raspberry Pi 4 Exploration}

    \author{Enrique~Antunano\\University~of~Washington\\enriantu@uw.edu\\https://github.com/enriquea02/uw}

    \markboth{EEP522A~Embedded and~Real~Time~Systems, A3~Configuration, February~2025}{}

    \maketitle

    \begin{abstract}

    \end{abstract}
    \section{Introduction}

    \section{Methods}

    \section{Results}

    \section{Discussion}

    \section{Conclusion}

    \nocite{*}
    \newpage

    \bibliographystyle{ieeetr}
    \bibliography{final_project.bib}

    \newpage
    \section{Acknowledgements}

    \newpage
    \appendices
    \section{Download WiringPi Library}

    \section{Setup RaspController Mobile Application}
    \emph{RaspController} is an \emph{iOS} app that allows for easy SSH access into your local Raspberry Pi 4 device.
    The \emph{RaspController} mobile app will be used to control the laser projector manually or drop new laser projection files into the Raspberry Pi 4 device.
    Unless port-fowarding is enabled, your mobile phone and Raspberry Pi 4 device must be on the same network, as this can be treated as any other SSH connection.

    As a future improvement for mass-production, this process should be changed to be more user-friendly, simpler, and less powerful for a regular user.
    At the moment, since the user is SSH-ing into the Raspberry Pi 4 operating system, there's a significant amount of room for error if the user does not know what they are doing.
    My suggestion for making this feature more user-friendly would be to file share via Bluetooth instead.
    For small-volume production, this is an easy path forward and reduces the complexity to the end-user and the risk that the user causes an error.
    As a prototype project though, having full control via a phone is very helpful for development.
    
    Follow the steps below for downloading and connecting to your Raspberry Pi 4 device:
    \begin{enumerate}
        \item Go to the app store, search for \emph{RaspController}, and download the application
        
        \item Navigate through Apple's privacy settings
        
        \item Open the RaspController app and select \emph{Add Device} 
        
        \item Fill in Raspberry Pi 4 device information in the \emph{Add Device} screen
        
        \includegraphics[width=2.5in]{add_device_raspcontroller_app.PNG}
        
        \item Select the added device
        
        \item Ran linux command to test SSH connection
        
        \begin{lstlisting}[frame=single]
            ifconfig
        \end{lstlisting}

        \includegraphics[width=2.5in]{ssh_shell_raspcontroller_app.PNG}
    \end{enumerate}


\end{document}