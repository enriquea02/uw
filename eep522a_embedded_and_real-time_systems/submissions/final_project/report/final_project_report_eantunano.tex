\documentclass[journal]{IEEEtran}
\usepackage{graphicx}
\usepackage{hyperref}
\usepackage{listings}
% Path is relative to final_project_report_eantunano.tex
\graphicspath{ {./images/}}

\begin{document}

    \title{Raspberry Pi 4 Exploration}

    \author{Enrique~Antunano\\University~of~Washington\\enriantu@uw.edu\\https://github.com/enriquea02/uw}

    \markboth{EEP522A~Embedded and~Real~Time~Systems, A3~Configuration, February~2025}{}

    \maketitle

    \begin{abstract}

    \end{abstract}
    \section{Introduction}

    \section{Methods}
    \subsection {Coding Standard}
    This project will use the LLVM Coding Standards. 
    For my decision, I pulled three coding standards: Google C++ Style Guide, LLVM Coding Standards, and Motor Industry Software Reliability Association (MISRA) C/C++.
    MISRA C/C++ will not be used for this projects because it is a purchased standard.
    MISRA's focus is on safety-critical applications, which I think is not applicable for my project.
    Even if my project were safety-critical, there is so little time to develop a working prototype that I care more about readability and consistency than safety-critical applications.
    For a mass-product, which I'll classify as 10,000+ units annually, or human-rated vehicle, then it would make sense to implement a safety-critical coding standard after the prototype phase.
    In my opinion, an engineering development unit would be a good candidate to implement this coding standard.

    I have decided to use the Google C++ Style Guide because it has a better website than the LLVM Coding Standards website.
    The coding standard only applies to new code that I have written for this project.
    It does not apply to source code added to my project from other github repositories.
    Source code will maintain whatever coding standard that it was written with and be used "as-is".
    Easier is entirely subjective. For myself, it is easier to read, navigate, and less verbose.
    The style guide is used for Google open-sourced projects, so it's widely used.
    My focus is on writing prototype code, so I don't want to sit down and read through LLVM Coding Standards right now.
    There may be more merit to the LLVM Coding Standards, but that decision may have to be for another day.

    Click {\href{https://google.github.io/styleguide/cppguide.html}{here}} to be taken to the Google C++ Style Guide.

    \subsection{Interfaces}
    The laser controller requires the following physical interfaces:
    \begin{itemize}
        \item GPIO pin
        \item SPI pins
        \item Power Pins
    \end{itemize}

    For this project, Raspian's \emph{spidev} Linux utility was used to setup and drive the SPI interfaces from the Raspberry Pi 4. 
    The Linux utility it limited to only two SPI interfaces, which was sufficient for this project.
    Two SPI interfaces are required by this project to drive the two galvo motors.
    One interface will drive the X-axis mirrors while the other will drive the Y-axis mirrors.
    
    Refer to the code below for checking if your system has the \emph{spidev} utility.

    \begin{lstlisting}[frame=single, basicstyle=\ttfamily\footnotesize, breaklines=true]
        ls -la /dev/
    \end{lstlisting}

    Search for a folder named \emph{spidev}.
    There may be numbers as part of the folder name and that's okay.
    Follow the instructions below if the \emph{spidev} folder does not exist.
    If not, read the screenshot below to install the \emph{spidev} library.

    \includegraphics[width=2.5in]{enable_spi_on_rpi4.png}

    Now search for the library again because you should see the following two folders after reboot.

    \includegraphics[width=2.5in]{spi4_installed_after_reboot.png}

    The laser controller requires the following digital interfaces:

    \section{Results}

    \section{Discussion}
    \subsection{Laser Projection Code}
    Vector laser projectors use an industry standard file format, set by the \emph{International Laser Display Association}, to convert images and videos into a laser display.
    Files are saved with the \emph{.ild} file handle.
    The file describes the X-Y commands and laser ON/OFF sequences which drive the mirror motors and laser diode respectively.

    There is software, such as Laserworld's \emph{Showeditor}, that can export files into a \emph{.ild} format.
    This project avoids the use of commercial software, such as \emph{Showeditor} for two reasons.
    First, their free tiers, if available at all, have very limited functionality.
    My research indicates that the free-tier from \emph{Showeditor} is sufficient for simple \emph{.ila} file exporting. 
    Any additional features will require a professional license though.
    For hobbyist purposes, \emph{Showeditor} seems like the only/best option.
    Second, my primary workstation only has Linux installed and the free-tier of \emph{Showeditor} only supports up to Windows Vista.
    Although I could install a VM or a dual-boot system, I would rather avoid installing an old version of Windows for the use of running one software package.
    As a final statement, I did try to run Showeditor using \emph{Wine} on my Ubuntu-based Linux machine, but \emph{Showeditor} failed to boot.

    Due to the issues highlighted, I found a workaround. 
    The github user, \emph{marcan}, has developed a {\href{https://github.com/marcan/openlase/blob/master/tools/svg2ild.py}{script}} that converts \emph{.svg} file types into \emph{.ild} file types.
    There are known bugs and issues with the script, code inefficiencies and possible missing RGB support, but as a development project those issues are fine. 
    My current project only works with one light source, red, so possible missing RGB support is not an issue.
    Plus, I have found another user who has implemented \emph{marcan's svg2ild.py} script onto a Raspberry Pi Zero and has a functional laser projector.
    Click {\href{https://github.com/phuid/laser_projector?tab=readme-ov-file#hw}{here}} to be taken to the github page of \emph{phuid}, who has implemented the \emph{svg2ild.py} script for use in their Raspberry Pi Zero.
    Since the project has been implemented by another user on a more resource constrained Raspberry Pi Zero, I am not worried about code inefficiencies with the \emph{svg2ild.py} script.

    To create a laser animation using a \emph{.svg} file, go to appendix \emph{Create Laser Animation}.

    \subsection{Discrete Electronics}
    Sourcing parts on a short notice, 3-5 weeks prior to March 17th, presented challenges with lead times.
    Long lead times forced me to substitute parts, such as a single dual-channel DAC with two single-channel DACs to drive the X and Y galvomotors respectively.
    Also, the long lead times not only limited my selection of parts, but limited where I could purchase those parts at such low volumes.


    Amazon was the best option due to their return policy, delivery speed, and having at least one option/substitute electronic component available.
    The drawbacks of using Amazon were, first, it was more expensive for equal or lower quality parts compared to vendors such as Digikey, Avnet, or AliExpress.
    Many online writers of at-home vector laser projectors sourced cheap parts from AliExpress, but none mentioned the long lead times for the components.
    The second drawback of Amazon was that many parts had sparse electrical characteristic documentation.
    For this project, before I switched from a stepper motor to a galvomotor, I was interested in the stepper motor selection on Digikey because they were better documented and were more precise. 
    Ultimately, I went with a stepper motor on Amazon that lacked documentation because the motor would arrive faster and the motor was 50\% cheaper. 
    One example was when I purchased a motor driver circuit from Amazon.
    The vendor had no workable documentation on their Amazon page, such as a pin out diagram.
    Before switching from a stepper motor to a galvomotor, which made the motor driver irrelevant for my project, I had to look up the datasheet of the TI DRV8825 microcontroller that was on the PCBA and assume all the datasheet information was still applicable.

    Another challenge that I faced was with handling substitute parts. 
    After extensive research, I found a set of vector laser projects that were well documented and were complete enough for me to use and implement in such a short period of time.
    Due to the time crunch of this project, I had to use substitute parts from the source documentation.
    The new parts forced me to do circuit modifications, write code to target different microcontrollers, and do additional research to verify that the Raspberry Pi 4 had enough pins and that I wouldn't fry the Raspberry Pi 4 when hooking up the substitute components.
    For this project, I had to substitute:
    \begin{itemize}
        \item Dual-channel DAC for two single-channel DACs
        \item XY Galvo-motors and motor driver boards
        \item Arduino with a Raspberry Pi 4
    \end{itemize} 

    The biggest pain was the substitute of the Dual-channel DAC for the two single-channel DACs. 
    The DACs are not as well in-sync as a single chip dual-channel DAC.
    The \emph{ABElectronic} code that I pulled was written for the dual-channel DACs, so the driver code had to be modified to drive the two DACs separately.
    Due to the timetable of this project, I dropped the more elegant single SPI interface solution, which would've used the chip-selects to communicate with DAC-X and DAC-Y separately.
    Implementation of the single SPI interface may not be difficult, but I was not sure how time-multiplexing the SPI bus would work.
    With most of the pins still free and a whole second set of Linux utility controlled SPI interface pins, I went with using both SPI interface resources.
    From an implementation standpoint, this is easier since I can drive both buses simultaneously without having to deal with a time-multiplexing solution that would still allow the X and Y mirrors to move in-sync.
     

    \section{Conclusion}

    \nocite{*}
    \newpage

    \bibliographystyle{ieeetr}
    \bibliography{final_project.bib}

    \newpage
    \section{Acknowledgements}
    The author would like to acknowledge the \emph{WiringPi} team for their GPIO access library for Raspberry Pi boards. \newline

    The author would like to acknowledge the \emph{RaspController} team for their remote access mobile application for Raspberry Pi boards.

    The author would like to acknowledge {\href{https://www.circuit-diagram.org/editor/}{Circuit Diagram}} for their free circuit design software. 

    \newpage
    \appendices
    \section{Download WiringPi Library}
    The library can be built on the device or a prebuilt library can be used. 
    For the Raspberry Pi 4 Model B using the Bullseye Debian in this project, the latest library in the WiringPi GitHub ({\href{https://github.com/WiringPi/WiringPi/releases/tag/3.14}{Release 3.14}}) will be used.

    In a Raspberry Pi SSH terminal, follow the code below to build the WiringPi library.
    \begin{lstlisting}[frame=single, basicstyle=\ttfamily\footnotesize, breaklines=true]
        # fetch the source
        sudo apt install git
        git clone https://github.com/WiringPi/WiringPi.git
        cd WiringPi

        # build the package
        ./build debian
        mv debian-template/wiringpi_3.14_armhf.deb .

        # install it
        sudo apt install ./wiringpi_3.14_armhf.deb
    \end{lstlisting}

    WiringPi was installed in the \emph{\$HOME} directory. 
    If the WiringPi library is installed in a different location in your Raspberry Pi system, then update laser controller scripts to reflect the location of the WiringPi library in your system.

    \section{Setup RaspController Mobile Application}
    \emph{RaspController} is an \emph{iOS} app that allows for easy SSH access into your local Raspberry Pi 4 device.
    The \emph{RaspController} mobile app will be used to control the laser projector manually or drop new laser projection files into the Raspberry Pi 4 device.
    Unless port-fowarding is enabled, your mobile phone and Raspberry Pi 4 device must be on the same network, as this can be treated as any other SSH connection.

    As a future improvement for mass-production, this process should be changed to be more user-friendly, simpler, and less powerful for a regular user.
    At the moment, since the user is SSH-ing into the Raspberry Pi 4 operating system, there's a significant amount of room for error if the user does not know what they are doing.
    My suggestion for making this feature more user-friendly would be to file share via Bluetooth instead.
    For small-volume production, this is an easy path forward and reduces the complexity to the end-user and the risk that the user causes an error.
    As a prototype project though, having full control via a phone is very helpful for development.
    
    Follow the steps below for downloading and connecting to your Raspberry Pi 4 device:
    \begin{enumerate}
        \item Go to the app store, search for \emph{RaspController}, and download the application
        
        \item Navigate through Apple's privacy settings
        
        \item Open the RaspController app and select \emph{Add Device} 
        
        \item Fill in Raspberry Pi 4 device information in the \emph{Add Device} screen
        
        \includegraphics[width=2.5in]{add_device_raspcontroller_app.PNG}
        
        \item Select the added device
        
        \item Ran linux command to test SSH connection
        
        \begin{lstlisting}[frame=single, basicstyle=\ttfamily\footnotesize, breaklines=true]
            ifconfig
        \end{lstlisting}

        \includegraphics[width=2.5in]{ssh_shell_raspcontroller_app.PNG}
    \end{enumerate}

    \section{Create Laser Animation}
    To create a source \emph{.svg} file, which will be converted to a \emph{.ila} file format within the Raspberry Pi 4, follow the guide below.

    Required Software:
    \begin{itemize}
        \item Blender 4.2 or newer
        \item Blender Add-on: Render: {\href{https://extensions.blender.org/add-ons/freestyle-svg-exporter/}{\emph{Freestyle SVG Exporter}}}
    \end{itemize} 

    Blender was selected because it is an open-source, free animation software that is available across most operating systems (Windows, MacOS, Linux).
    The Blender community is fairly large, so there is a sufficient amount of community support.
    The following {\href{https://youtu.be/PrlK5Y74sR8?si=fridcX0fLfz2cV0W}{YouTube}} video was used as a reference to understand how to generate \emph{.svg} file types. 

    \begin{enumerate}
        \item Download Blender software
        \item Download Freestyle SVG Exporter
        \item Follow on-screen instructions for download and select to enable the extension
        \item Follow YouTube video highlighted above for generating a \emph{.svg} file type, except change frame rate to 24fps
        \item The \emph{.svg} file will be generated in the output folder location that was specified 
    \end{enumerate}

    Use this YouTube video, {\href{https://youtu.be/CBJp82tlR3M?si=AnMmnkilE5h2m_C9}{Animation for Beginners}}, to generate an animation for the laser projector.

    \section{Schematics}
    This appendix covers the electrical schematics necessary to execute this project.
    Relevant datasheet material will also be included here.

    \subsection{Raspberry Pi 4}
    \subsection{DAC}
    Refer to Microchip's \emph{MCP4901/4911/4921 8/10/12-Bit Voltage Output Digital-to-Analog Converter with SPI Interface} document listed in the references for the source material.

    The Raspberry Pi 4 drive strength will have to be limited to 2mA to match the DAC's input current electrical characteristics.
    One concern reading the datasheet was that the current at the supply pins were rated to an absolute maximum of 50mA.
    Since the only DACs available, with such short notice on Amazon. were single channel DACs, I needed to purchase two single-channel DACs.
    Each DAC channel drives a single galvomotor. 
    Laser projectors require two galvomotors, one for the X direction and another for the Y direction.
    What concerned me was whether the 5V pins on the Raspberry Pi 4 Model B could supply 100mA (50mA * 2 DACs).
    Reading multiple forums, the current on the 5V rails on the Raspberry Pi 4 are primarily limited by the Raspberry Pi 4 power supply.
    The power supply used in this design is rated for 3.5A, so 100mA is well below that threshold, even accounting for the Raspberry Pi 4's own current draw and any other components. \newline 

    \includegraphics[width=2.5in]{mcp4921_electrical_characteristics.png}

    The gain formula with respect to the reference voltage is listed in the image below. \newline

    \includegraphics[width=2.5in]{mcp4921_gain_formula.png}

    Referring the pinout descriptions below, separate nCS, SCK, and SDI lines will be driven from the Raspberry Pi 4 to the respective X and Y DACs.
    Both DACs will need to be driven simultaneously and I have spare pins on the Raspberry Pi 4, thus each DAC will have their own set of SPI inputs.
    nLDAC will be tied to GND, referenced to the Raspberry Pi 4, since $V_{out}$ control can be set on the rising edge of nCS.
    Tying nLDAC to low also saves the use of two pins on the Raspberry Pi 4.
    $V_{REF}$ will be set to 2.0143V because it is relatively close to the typical 2.038V reference voltage that is listed in the Microchip documentation.
    I can use a simple voltage divider on the 3.3V power rail pins of the Raspberry Pi 4 to get 2.0143V.
    The documentation is unclear whether $V_{DD}$ an $V_{REF}$ need to be related.
    Additionally, it is unclear whether the input SPI pins need to be related to $V_{DD}$ or $V_{REF}$.
    For now, I'm moving forward with the risk since I think it's low.
    The Raspberry Pi 4 has an OCP circuit that I think will protect the Raspberry Pi 4.
    Finally, the voltage divider for $V_{REF}$ uses high resistance vales to limit the current flow to the DAC. \newline

    \includegraphics[width=2.5in]{mcp4921_pin_descriptions.png}

    Final note on the DAC circuit, a bypass capacitor and a high-frequency noise filtering capacitor were placed on the 5V rail as recommended by the datasheet. \newline

    \includegraphics[width=2.5in]{mcp4921_psu_and_dig_intf.png}

    \subsection{ILDA Op Amp}


\end{document}