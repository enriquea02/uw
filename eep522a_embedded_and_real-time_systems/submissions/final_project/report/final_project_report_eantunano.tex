\documentclass[journal]{IEEEtran}
\usepackage{graphicx}
\usepackage{hyperref}
\usepackage{listings}
% Path is relative to final_project_report_eantunano.tex
\graphicspath{ {./images/}}

\begin{document}

    \title{Raspberry Pi 4 Exploration}

    \author{Enrique~Antunano\\University~of~Washington\\enriantu@uw.edu\\https://github.com/enriquea02/uw}

    \markboth{EEP522A~Embedded and~Real~Time~Systems, A3~Configuration, February~2025}{}

    \maketitle

    \begin{abstract}

    \end{abstract}
    \section{Introduction}

    \section{Methods}

    \section{Results}

    \section{Discussion}
    \subsection{Laser Projection Code}
    Vector laser projectors use an industry standard file format, set by the \emph{International Laser Display Association}, to convert images and videos into a laser display.
    Files are saved with the \emph{.ild} file handle.
    The file describes the X-Y commands and laser ON/OFF sequences which drive the mirror motors and laser diode respectively.

    There is software, such as Laserworld's \emph{Showeditor}, that can export files into a \emph{.ild} format.
    This project avoids the use of commercial software, such as \emph{Showeditor} for two reasons.
    First, their free tiers, if available at all, have very limited functionality.
    My research indicates that the free-tier from \emph{Showeditor} is sufficient for simple \emph{.ila} file exporting. 
    Any additional features will require a professional license though.
    For hobbyist purposes, \emph{Showeditor} seems like the only/best option.
    Second, my primary workstation only has Linux installed and the free-tier of \emph{Showeditor} only supports up to Windows Vista.
    Although I could install a VM or a dual-boot system, I would rather avoid installing an old version of Windows for the use of running one software package.
    As a final statement, I did try to run Showeditor using \emph{Wine} on my Ubuntu-based Linux machine, but \emph{Showeditor} failed to boot.

    Due to the issues highlighted, I found a workaround. 
    The github user, \emph{marcan}, has developed a {\href{https://github.com/marcan/openlase/blob/master/tools/svg2ild.py}{script}} that converts \emph{.svg} file types into \emph{.ild} file types.
    There are known bugs and issues with the script, code inefficiencies and possible missing RGB support, but as a development project those issues are fine. 
    My current project only works with one light source, red, so possible missing RGB support is not an issue.
    Plus, I have found another user who has implemented \emph{marcan's svg2ild.py} script onto a Raspberry Pi Zero and has a functional laser projector.
    Click {\href{https://github.com/phuid/laser_projector?tab=readme-ov-file#hw}{here}} to be taken to the github page of \emph{phuid}, who has implemented the \emph{svg2ild.py} script for use in their Raspberry Pi Zero.
    Since the project has been implemented by another user on a more resource constrained Raspberry Pi Zero, I am not worried about code inefficiencies with the \emph{svg2ild.py} script.

    To create a laser animation using a \emph{.svg} file, go to appendix \emph{Create Laser Animation}.

    \section{Conclusion}

    \nocite{*}
    \newpage

    \bibliographystyle{ieeetr}
    \bibliography{final_project.bib}

    \newpage
    \section{Acknowledgements}
    The author would like to acknowledge the \emph{WiringPi} team for their GPIO access library for Raspberry Pi boards. \newline

    The author would like to acknowledge the \emph{RaspController} team for their remote access mobile application for Raspberry Pi boards.

    \newpage
    \appendices
    \section{Download WiringPi Library}
    The library can be built on the device or a prebuilt library can be used. 
    For the Raspberry Pi 4 Model B using the Bullseye Debian in this project, the latest library in the WiringPi GitHub ({\href{https://github.com/WiringPi/WiringPi/releases/tag/3.14}{Release 3.14}}) will be used.

    In a Raspberry Pi SSH terminal, follow the code below to build the WiringPi library.
    \begin{lstlisting}[frame=single, basicstyle=\ttfamily\footnotesize, breaklines=true]
        # fetch the source
        sudo apt install git
        git clone https://github.com/WiringPi/WiringPi.git
        cd WiringPi

        # build the package
        ./build debian
        mv debian-template/wiringpi_3.14_armhf.deb .

        # install it
        sudo apt install ./wiringpi_3.14_armhf.deb
    \end{lstlisting}

    WiringPi was installed in the \emph{\$HOME} directory. 
    If the WiringPi library is installed in a different location in your Raspberry Pi system, then update laser controller scripts to reflect the location of the WiringPi library in your system.

    \section{Setup RaspController Mobile Application}
    \emph{RaspController} is an \emph{iOS} app that allows for easy SSH access into your local Raspberry Pi 4 device.
    The \emph{RaspController} mobile app will be used to control the laser projector manually or drop new laser projection files into the Raspberry Pi 4 device.
    Unless port-fowarding is enabled, your mobile phone and Raspberry Pi 4 device must be on the same network, as this can be treated as any other SSH connection.

    As a future improvement for mass-production, this process should be changed to be more user-friendly, simpler, and less powerful for a regular user.
    At the moment, since the user is SSH-ing into the Raspberry Pi 4 operating system, there's a significant amount of room for error if the user does not know what they are doing.
    My suggestion for making this feature more user-friendly would be to file share via Bluetooth instead.
    For small-volume production, this is an easy path forward and reduces the complexity to the end-user and the risk that the user causes an error.
    As a prototype project though, having full control via a phone is very helpful for development.
    
    Follow the steps below for downloading and connecting to your Raspberry Pi 4 device:
    \begin{enumerate}
        \item Go to the app store, search for \emph{RaspController}, and download the application
        
        \item Navigate through Apple's privacy settings
        
        \item Open the RaspController app and select \emph{Add Device} 
        
        \item Fill in Raspberry Pi 4 device information in the \emph{Add Device} screen
        
        \includegraphics[width=2.5in]{add_device_raspcontroller_app.PNG}
        
        \item Select the added device
        
        \item Ran linux command to test SSH connection
        
        \begin{lstlisting}[frame=single, basicstyle=\ttfamily\footnotesize, breaklines=true]
            ifconfig
        \end{lstlisting}

        \includegraphics[width=2.5in]{ssh_shell_raspcontroller_app.PNG}
    \end{enumerate}

    \section{Create Laser Animation}
    To create a source \emph{.svg} file, which will be converted to a \emph{.ila} file format within the Raspberry Pi 4, follow the guide below.

    Required Software:
    \begin{itemize}
        \item Blender 4.2 or newer
        \item Blender Add-on: Render: {\href{https://extensions.blender.org/add-ons/freestyle-svg-exporter/}{\emph{Freestyle SVG Exporter}}}
    \end{itemize} 

    Blender was selected because it is an open-source, free animation software that is available across most operating systems (Windows, MacOS, Linux).
    The Blender community is fairly large, so there is a sufficient amount of community support.
    The following {\href{https://youtu.be/PrlK5Y74sR8?si=fridcX0fLfz2cV0W}{YouTube}} video was used as a reference to understand how to generate \emph{.svg} file types. 

    \begin{enumerate}
        \item Download Blender software
        \item Download Freestyle SVG Exporter
        \item Follow on-screen instructions for download and select to enable the extension
        \item Follow YouTube video highlighted above for generating a \emph{.svg} file type, except change frame rate to 24fps
        \item The \emph{.svg} file will be generated in the output folder location that was specified 
    \end{enumerate}

    Use this YouTube video, {\href{https://youtu.be/CBJp82tlR3M?si=AnMmnkilE5h2m_C9}{Animation for Beginners}}, to generate an animation for the laser projector.

    

\end{document}